\begin{frame}
	\frametitle{Discretization and reconstruction of signals}
	The use of digital logic and computers to calculate a control action for a continuous system introduces the operation of sampling.\\
	\medskip
	Samples are taken from the continuous signals and used in the computer to calculate the controls to be applied.\\
	\medskip
	The role of sampling and the conversion from continuous to discrete and vice versa are important to the understanding of the complete response of digital control.
	\begin{figure}
		\includegraphics[width=.6\linewidth]{discretization}
	\end{figure}
	%source: http://cnx.org/contents/42c54126-3078-417f-972e-dcf5c41556a4@3.1:6/Software-Receiver-Design
\end{frame}

\section{Analysis of the sample and hold}

\begin{frame}
	\frametitle{Analysis of the sample and hold}
	\vspace{-6ex}
	To get samples of a continuous signal, we use an analog-to-digital converter. The conversion always takes a non-zero time, often this time is significant with respect to the sample period.\\
	\medskip
	To give the computer an accurate representation of the signal exactly at the sampling instants kT, the converter is preceded by a sample-and-hold circuit.\\
	The sample-and-hold will take the impulses that are produced by the mathematical sampler and produce the piecewise constant output of the device.
\end{frame}

\begin{frame}
	\frametitle{Sampling operation}
	\vspace{-4ex}
	The sampling operation is represented by impulse modulation. Its role is to give a mathematical representation of taking periodic samples from $r(t)$ to produce $(kT)$. \\
	\medskip
	The sampler takes as input $r(t)$ and returns as output
	\bigskip
	$r^*(t)=\sum_{k=-\infty}^{\infty} r(t)\delta(t-kT)$.\\
	The Laplace transform of $r^*(t)$ can be computed as\\
	\vspace{-2ex} 
	\begin{align*} 
	\mathcal{L}\{r^*(t)\} = \int_{-\infty}^{\infty} r^*(\tau)e^{-s\tau} d\tau = \int_{-\infty}^{\infty} \sum_{k=-\infty}^{\infty} r(\tau)\delta(\tau-kT)e^{-s\tau}d\tau 
	\end{align*}\\
\end{frame}

\begin{frame}
	\frametitle{Sampling operation}
	Using $\int_{-\infty}^{\infty} f(t)\delta(t-a)dt = f(a)$ we obtain \\ 
	\begin{center}
		$\mathcal{L}\{r^*(t)\} = R^*(s) = \sum_{k=-\infty}^{\infty} r(kT)e^{-skT}$
	\end{center}
	If the signal r(t) is shifted a small amount, then different samples will be selected by the sampling process for the output, proving that sampling is not a time-invariant process.
	\begin{figure}
		\includegraphics[width=0.6\linewidth]{sampled_signal}
	\end{figure}
	%source: https://commons.wikimedia.org/wiki/File:Sampled.signal.svg
\end{frame}

\begin{frame}
	\frametitle{Hold operation}
	The hold operation is represented as a linear filter. It is defined by means whereby the impulses are extrapolated to the piecewise constant signal $r_h(t) = r(kT)$ with $kT \leq t < (k+1)T$.\\
	\medskip
	A general technique is to use a polynomial fit to the past samples.\\
	If the extrapolation is done by a constant (a zero-order polynomial), then the extrapolator is called a zero-order hold and its transfer function is $ZOH(s)$. 
	\begin{figure}
		\includegraphics[width=0.55\linewidth]{sample_and_hold}
	\end{figure}
	%source: https://en.wikipedia.org/wiki/Sample_and_hold#/media/File:Zeroorderhold.signal.svg
\end{frame}

\begin{frame}
	\frametitle{Zero-order hold}
	We compute the transfer function as the transform of its impulse response. \\
	\medskip
	If $r^*(t)=\delta(t)$ then $r_h(t)$ is a pulse of height 1 and duration T:\\
	\begin{center}
		$r_h(t) = 1(t) - 1(t-T)$\\
	\end{center}
	with Laplace transform\\
	\begin{center}
		$ZOH(s)=\mathcal{L}\{p(t)\} = \int_{0}^{\infty} [1(t)-1(t-T)]e^{-s\tau}d\tau = \frac{1-e^{-sT}}{s}$
	\end{center}
	\begin{figure}
		\includegraphics[width=0.5\linewidth]{zoh}
	\end{figure}
	%source: http://www.audiorecordingschool.com/blog/chris-montgomery-explains-digital-audio/
\end{frame}


\section{Fourier transform}

\begin{frame}
	\frametitle{Fourier transform}
	\vspace{-2ex}
	\large{\[
	\equalto{f(t)}{\int_{-\infty}^{\infty} e^{j\omega t} d\omega} \Leftrightarrow \equalto{F(j\omega)}{\int_{-\infty}^{\infty} f(\tau) e^{-j \omega \tau} d \tau = |F(j \omega) e^{j \phi(\omega)}|}
	\]}\\
	\begin{figure}
		\includegraphics[width=0.6\linewidth]{fourier_examples}
	\end{figure}
\end{frame}

\begin{frame}
	\frametitle{Fourier transform: properties}
	\begin{itemize}
		\item \textbf{Linearity} \\
		\vspace{-3ex}
		\begin{align*}
		& \begin{cases}
		f_1(t) \leftrightarrow F_1(j\omega)\\
		f_2(t) \leftrightarrow F_2(j\omega)
		\end{cases} && \Rightarrow \quad a f_1(t) + b f_2(t) \leftrightarrow a F_1(j\omega) + b F_2(j \omega)
		\end{align*}
		\item \textbf{Time-scaling} \\
		\medskip
		$f(a t) \leftrightarrow (\frac{1}{|a|}) F(\frac{j \omega}{a})$
		\medskip
		\item \textbf{Translation/Time-shifting} \\
		\medskip
		$f (t - t_0) \leftrightarrow e^{-j \omega t_0} F(j\omega)$
		\medskip
		\item \textbf{Modulation/Frequency-shifting} \\
		\medskip
		$e^{j \omega_0 t} f(t) \leftrightarrow F(j (\omega - \omega_0))$
	\end{itemize}
\end{frame}

\begin{frame}
	\frametitle{Fourier transform: properties}
	\begin{itemize}
		\item \textbf{Reciprocity} \\
		\medskip
		$F(-jt) \leftrightarrow  2 \pi f(\omega)$
		\medskip
		\item \textbf{Derivative in t} \\
		\medskip
		$\frac{df(t)}{dt} \leftrightarrow j\omega F(j\omega) \qquad \qquad \quad \frac{d^nf(t)}{dt^n} \leftrightarrow (j\omega)^n F(j\omega)$
		\medskip
		\item \textbf{Derivative in $\omega$} \\
		\medskip
		$(-jt)^n f(t) \leftrightarrow \frac{d^n F(j\omega)}{d\omega^n} \qquad \quad \frac{f(t)}{-jt} \leftrightarrow \int_{-\infty}^\infty F(j\Omega) d\Omega \> if \> f(0) = 0$
		\medskip
		\item \textbf{Convolution} \\
		\medskip
		$y(t) = h(t) * u(t) \leftrightarrow Y(j\omega) = H(j\omega) U(j\omega)$\\
		$v(t) = h(t)u(t) \leftrightarrow V(j\omega) = \frac{1}{2\pi} H(j\omega)*U(j\omega)$
	\end{itemize}
\end{frame}

\section{Spectrum of a sampled signal}

\begin{frame}
	\frametitle{Spectrum of a sampled signal}
	\vspace{-5ex}
	To get further insight into the process of sampling, we use an alternative representation of the transform of $r^*(t)$ using Fourier analysis.\\
	\medskip
	$r^*(t)$ is a product of $r(t)$ and a train of impulses. The latter series is periodic and can be represented by a Fourier series\\
	\begin{center}
		$\sum_{k=-\infty}^{\infty} \delta(t-kT) = \sum_{n=-\infty}^{\infty} C_ne^{j(2\pi n/T)}t$,
	\end{center}
	where Fourier coefficients $C_n$ are given by:\\
	\begin{center}
		$C_n=\frac{1}{T}\int_{-T/2}^{T/2} \sum_{k=-\infty}^{\infty} \delta(t-kT)e^{-jn(2\pi t/T)}dt$.
	\end{center}
	
\end{frame}

\begin{frame}
	\frametitle{Spectrum of a sampled signal}
	\vspace{-10ex}
	The only term in the sum of impulses that is in the range of the integral is the $\delta(t)$ at the origin, so the integral reduces to \\
	\begin{center}
		$C_n=\frac{1}{T}\int_{-T/2}^{T/2}\delta(t)e^{-jn(2\pi t/T)}dt=\frac{1}{T}$
	\end{center}
	We derived the Fourier series of the sum of impulses\\
	\begin{center}
		$\sum_{k=-\infty}^{\infty} \delta(t-kT)=\frac{1}{T}\sum_{n=-\infty}^{\infty} e^{j(2\pi n/T)t}$
	\end{center}
	We define $\omega_s = \frac{2\pi}{T}$ as the sampling frequency (rad/s).\\
\end{frame}

\begin{frame}
	\frametitle{Spectrum of a sampled signal}
	We take the Laplace transform of the output of the sampler,
	\vspace{-2ex}
	\begin{center}
		\begin{equation}
			\begin{split}
			R^*(s)& = \int_{-\infty}^{\infty} r(t) \Big\{ \frac{1}{T} \sum_{n=-\infty}^{\infty} e^{jn\omega_st} \Big\} e^{-st} dt\\
			& = \frac{1}{T} \sum_{n=-\infty}^{\infty} \int_{-\infty}^{\infty} r(t) e^{jn\omega_st}e^{-st}dt\\
			& = \frac{1}{T} \sum_{n=-\infty}^{\infty} \int_{-\infty}^{\infty}r(t) e^{-(s-jn\omega_s)t} dt.
			\end{split}
		\end{equation}
	\end{center}
	Since the integral is the Laplace transform of $r(t)$ with only a change of variable where the frequency goes, the result can be written as:\\
	\begin{center}
		$R^*(s)=\frac{1}{T}\sum_{n=-\infty}^{\infty}R(s-jn\omega_s)$.
	\end{center}
\end{frame}

\subsection{Aliasing}

\begin{frame}
	\frametitle{Aliasing}
	Aliasing is	an effect that causes different signals to become indistinguishable when sampled. Frequencies that are too high to be sampled are folded onto lower frequencies. We cannot distinguish between them based on their samples alone.\\
	\medskip
	Take for example the image below.\\
	The red sine wave is being sampled at just over it's bandwidth, however the blue sine wave will be recreated as it also fit's all data points and is within the expected bandwidth.
	\begin{figure}
		\includegraphics[width=0.8\linewidth]{aliasing}
	\end{figure}
	%source: https://en.wikipedia.org/wiki/Aliasing#/media/File:AliasingSines.svg
\end{frame}

\begin{frame}
	\frametitle{Aliasing}
	As a direct result of the sampling operation, when data are sampled at a frequency $\frac{2\pi}{T}$, the total harmonic content at a given frequency $\omega_1$ is to be found not only from the original signal at $\omega_1$, but also from all those frequencies that are aliases of $\omega_1$, namely $\omega_1 + n2\pi/T = \omega_1+n\omega_s$.\\
	\medskip
	The errors caused by aliasing can be very severe if a substantial quantity of high-frequency components is contained in the signal to be sampled.\\
	\medskip
	To minimize this error, the sampling operation is preceded by a low-pass antialias filter that will remove all spectral content above the half-sampling frequency ($\pi/T$).
\end{frame}

\subsection{Sampling theorem}

\begin{frame}
	\frametitle{Sampling theorem}
	If all content above the half-sampling frequency is removed, no aliasing is introduced by sampling. Also the signal spectrum is not distorted, even though it is repeated endlessly, centered at $n2\pi/T$.\\
	\medskip
	This critical frequency, $\pi/T$, is called the \textbf{Nyquist frequency}. Band-limited signals that have no components above the Nyquist frequency are represented unambiguously by their samples. \\
	\medskip
	This is the \textbf{sampling theorem}: One can recover a signal from its samples if the sampling frequency ($\omega_s=2\pi/T$) is at least twice the highest frequency ($\pi/T$) in the signal. This maximum frequency is also called the \textbf{bandwidth}.
\end{frame}

\begin{frame}
	\frametitle{Sampling theorem}
	\begin{columns}
		\column{.5\textwidth}
		The signal can be fully reconstructed if there are no overlaps in the frequency domain.\\
		If the sampling frequency is too low then information will be lost (overlap).\\
		If the sampling frequency is at least twice the bandwidth B, then the signal can be reconstructed without a problem (no overlap).\\
		\medskip
		Sampling frequence $f_s \geq 2 B$
		\column{.5\textwidth}
		\begin{figure}
			\includegraphics[width=1\linewidth]{nyquist}
		\end{figure}
	\end{columns}
\end{frame}

\subsection{Hidden oscillations}

\begin{frame}
	\frametitle{Hidden oscillations}
	\vspace{-15ex}
	There is the possibility that a signal could contain some frequencies that the samples do not show at all. \\
	Such signals, when they occur in digital control systems, are called \textbf{hidden oscillations}. \\
	They can only occur at multiples of the Nyquist frequency ($\pi/T$).
\end{frame}

\section{Data extrapolation (reconstruction)}

\begin{frame}
	\frametitle{Reconstruction}
	\begin{columns}
		\column{0.4\textwidth}
		\underline{\smash{Sampling theorem}}: \textit{under the right conditions} it is possible to recover a signal from its samples.\\
		\medskip
		The figure to the right shows the spectrum of $R(j\omega)$. It is contained in the low-frequency part of $R^*(j\omega)$. Therefore, to recover $R(j\omega)$ we need to process $R^*(j\omega)$ through a low-pass filter and multiply by T.\\
		\column{0.6\textwidth}
		\vspace{-4ex}
		\begin{figure}
			\includegraphics[width=1.1\linewidth]{reconstruction}
		\end{figure}
	\end{columns}
\end{frame}

\begin{frame}
	\frametitle{Reconstruction}
	If $R(j\omega)$ has zero energy for frequencies in the bands above the Nyquist frequency, in other words R is band-limited, then an ideal low-pass filter with gain T for $-\pi/T \leq \omega \leq \pi/T$ and zero elsewhere would recover $R(j\omega)$ from $R^*(j\omega)$ exactly.\\
	\medskip
	If we define the ideal low-pass filter characteristic as $L(j\omega)$, we have:
	\begin{center}
		$R(j\omega)=L(j\omega)R^*(j\omega)$.
	\end{center}
	The signal $r(t)$ is the inverse transform of $R(j\omega)$. Because $R(j\omega)$ is the \textit{product} of two transforms, its inverse transform $r(t)$ is the \textit{convolution} of the time functions $\ell(t)$ and $r^*(t)$.\\
	\begin{center}
		$r(t) = l(t) * r^*(t)$
	\end{center}
\end{frame}

\begin{frame}
	\frametitle{Ideal low-pass filter}
	The form of the filter impulse response can be computed using this definition\\
	\vspace{-2ex}
	\begin{columns}
		\column{0.5\textwidth}
		\begin{equation}
		\begin{split}
		\ell(t) & = \frac{1}{2\pi} \int_{-\pi/T}^{\pi/T}Te^{j\omega t}d\omega\\
		& = \frac{T}{2\pi} \frac{e^{j\omega t}}{jt} \Big|_{-\pi/T}^{\pi/T}\\
		& = \frac{T}{2\pi jt}(e^{j(\pi t/T)}-e^{-j(\pi t/T)})\\
		& = \frac{sin(\pi t/T)}{\pi t/T}\\
		& \triangleq sinc\frac{\pi t}{T} \nonumber
		\end{split}
		\end{equation}
		\column{0.6\textwidth}
		\vspace{-2ex}
		\begin{figure}
			\includegraphics[width=1.1\linewidth]{sinc}
		\end{figure}
		%source: https://en.wikipedia.org/wiki/Rectangular_function#/media/File:Sinc_function_(normalized).svg
	\end{columns}
	The sinc functions are the interpolators that fill in the time gaps between samples with a signal that has no frequencies above $\pi/T$.
\end{frame}

\begin{frame}
	\frametitle{Reconstruction}
	Using the previous equations, we find:\\
	\medskip
	$r(t)=\int_{-\infty}^{\infty} r(\tau)\sum_{k=-\infty}^{\infty} \delta(\tau-kT)sinc\frac{\pi(t-\tau)}{T}d\tau$.\\
	\medskip
	Using the shifting property of the impulse, we have:\\
	 \[
	 \boxed{r(t)=\sum_{k=-\infty}^{\infty} r(kT)sinc\frac{\pi(t-kT)}{T}}
	 \]
	This equation is a constructive statement of the sampling theorem.  There is one disadvantage. Because $\ell(t)$ is nonzero for $t < 0$, this filter is noncausal. $\ell(t)$ starts at $t=-\infty$ while the impulse that triggers it does not occur until $t=0$. The noncausality can be overcome by adding a phase lag, $e^{-j\omega \lambda}$, to $L(j\omega)$, which adds a delay to the filter and to the signals processed through it.
\end{frame}

\begin{frame}
	\frametitle{Zero-order hold}
	The transfer function of the zero-order hold was introduced as\\
	\vspace{-1ex}
	\begin{center}
		$ZOH(j\omega)=\frac{1-e^{-j\omega T}}{j\omega}$.\\
	\end{center}
	\vspace{-1ex}
	We express this function in magnitude and phase form, to discover the frequency properties of $ZOH(j\omega)$.\\
	\medskip
	We factor out $e^{-j\omega T/2}$ and multiply and divide by $2j$:\\
	\vspace{-4ex}
	\begin{equation}
	\begin{split}
		ZOH(j\omega) &= e^{-j\omega T/2}\Big\{\frac{e^{j\omega T/2}-e^{-j\omega T/2}}{2j}\Big\} \frac{2j}{j\omega}\\
		& = Te^{-j\omega T/2} \frac{sin(\omega T/2)}{\omega T/2}\\
		& = e^{-j\omega T/2} T sinc(\omega T/2)
	\end{split} \nonumber
	\end{equation}
	\vspace{-1ex}
\end{frame}

\begin{frame}
	\frametitle{Zero-order hold}
	The magnitude function is\\
	\vspace{-1ex}
	\begin{center}
		$|ZOH(j\omega)| = T\Big|sinc\frac{\omega T}{2} \Big|$
	\end{center}
	\vspace{-1ex}
	and the phase is\\
	\vspace{-1ex}
	\begin{center}
		$\angle ZOH(j\omega)=\frac{-\omega T}{2}$
	\end{center}
	\vspace{-1ex}
	plus the $180^\circ$ shifts where the sinc function changes sign.\\
	\medskip
	Thus the effect of the zero-order hold is to introduce a phase shift of $\omega T/2$ (a time delay of $T/2$ seconds) and to multiply the gain by a function with the magnitude of $sinc(\omega T/2)$.
\end{frame}

