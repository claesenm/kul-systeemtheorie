\section{Introduction}

\begin{frame}
	\frametitle{Definitions}
		The main objective of this chapter: design and compensation of single-input-sigle-output linear time-invariant control systems.
		\begin{itemize}
			\item Compensation: the modification of the system dynamics to satisfy the given specifications.
			\item Specifications (transient response and steady-state requirements): given before the design.
			\item Design by root-locus method: making a new root locus by adding poles and zeros to the system's open-loop transfer function.
			\item Compensator: an other system inserted in parallel or in cascade with the system for the purpose of satisfying the specifications of the original system (e.g. lead, lag, lag-lead compensator's or PID controllers).
		\end{itemize}
\end{frame}

\begin{frame}
	\frametitle{Compensators}
	A sinusoidal input is applied to the input of a network. We got a:
	\begin{itemize}
		\item lead network: if the steady-state output has a phase lead.
		\item lag network: if the steady-state output has a phase lag.
		\item lag-lead network: if we have phase lag and phase lead in the output but in different frequentcy regions (lag when the input has low frequency and lead in high frequentcy). 
	\end{itemize}
	The amount of lag/lead is a function of the frequency. \vspace{4mm}
	
	A compensator with characteristic of a lead network, lag network, or lag-lead network is called a lead compensator, lag compensator, or lag-lead compensator. 
	
	\textbf{Remark}: with trail and error we find the optimal compensator
\end{frame}

\section{General design}

\begin{frame}
	\frametitle{Controllers}
		Scetching the problem:
		\begin{enumerate}
			\item We got a plant that not achieve all the specifications (and we cannot change the parameters of the original system).
			\item We have to change other parameters such that the system will achieve all the specifications.
			\item These other parameters can be changed by changing the root loci of the closed-loop system. 
			\item So we search the root loci of a compensator such that the overall system achieve the specifications. 
			\item We let the original system interact (cascade or parallel) with the compensator.
		\end{enumerate}
		\vspace{3mm}
		
		\textbf{Remark}: we discuss only continu time systems. 
\end{frame}

\begin{frame}
	\frametitle{Root locus approach}
	\underline{The method}:\\
	Given: the root loci of the open-loop system (this are the parameters).\\
	Now, the method consist of graphical determining the root loci of the closed-loop system.\vspace{3mm}

	\underline{Example}: 
	We want a certain gain of a system:
	\begin{itemize}
		\item The system is not stable at that gain.
		\item We have to change the root loci (and make the system stable) by designing a compensator
		\item Now, we got different root loci and the system is stable at the certain gain. So, the specifications are achieved.
	\end{itemize}
	\vspace{2mm}
	
	In \textbf{general}: we want to have the root loci of the system on the good locations.
\end{frame}

\begin{frame}
	\frametitle{Addition of poles/zeros}
	\underline{Addition of poles}:\\
	\begin{itemize}
		\item Pulls the root locus to the right.
		\item Lowers the system's relative stability.
		\item Slows down the setting of the response.
	\end{itemize}
	\vspace{3mm}
	
	\underline{Addition of zeros}:
	\begin{itemize}
		\item Pulls the root locus to the left.
		\item Increase the system's relative stability.
		\item Speeds up the settling of the response.
		\item Speeds up the transient response.
		\item Increases the anticipation of the system. 
	\end{itemize}
\end{frame}

\section{Lead copensations}

\begin{frame}
	\frametitle{Lead compensators}
		There are 3 ways to make a \underline{lead compensator}:
		\begin{enumerate}
			\item Electronic networks using operational amplifiers.
			\item Electrical RC networks.
			\item Mechanical spring-dashpot systems.
		\end{enumerate}
		\vspace{3mm}
		
		When to use the \underline{root locus approach} for lead compensators:\\
		$\rightarrow$ When the specifications are given in terms of time-domain quantities. Examples:
		\begin{itemize}
			\item damping ratio
			\item rise time
			\item setting time
			\item maximum overshoot
			\item undamped natural frequency
		\end{itemize}
\end{frame}

\begin{frame}
	\frametitle{Lead compensators}
	Steps to make a lead compensator (in cascade with the original system):
	\begin{enumerate}
		\item Determine the locations of the desired dominant poles (from the specifications).
		\item Draw the root locus of the uncompensated system. If we change the gain, then the poles will change. If we can achieve the good locations of the poles on this way, then there is no need for a lead compensator. Else, if we cannot achieve the good locations, then we have to make a lead compensator. 
		\item Assume the lead compensator of this form: \\
		$C(s)=K_c \alpha\frac{Ts+1}{\alpha Ts+1}= K_c\frac{s+\frac{1}{T}}{s+\frac{1}{\beta T}}$ with $0<\alpha<1$.
	\end{enumerate}
\end{frame}

\begin{frame}
	\frametitle{Lead compensators}
		$C(s)=K_c \alpha\frac{Ts+1}{\alpha Ts+1}= K_c\frac{s+\frac{1}{T}}{s+\frac{1}{\beta T}}$ with $0<\alpha<1$\\
		\begin{itemize}
			\item Determining $\alpha$ and T:\\
			$\alpha$ and T are determined by the angle $\phi$. This is the angle $\angle(C(s))$ must be equal to the difference between the angle of the desired pole and the angle of the original transfer function. \\
			Result: the closed-loop pole has the same angle as the desired angle (angle of the desired pole). (And by determine T and $\alpha$ we determine the poles and zeros of the compensator.)
			\item Determine $K_c$ (=the open-loop gain) from the magnitude specifications.\\
			Result: the closed-loop pole has the desired magnitude. 
		\end{itemize}
		\textbf{Remark}: if there is some freedom about the parameter $\alpha$, then take $\alpha$ as high as possible. 
\end{frame}

\section{Lag compensation}

\begin{frame}
	\frametitle{Lag compensators}
		 \textbf{Problem}: A system that satisfy the desired transient response but don't satisfy the steady-state. \\
		 $\rightarrow$ We got to compensate the system in such a way that it also satisfy the steady-state specifications (we have to make a lag compensator in cascade with the original system).
		 \vspace{4mm}
		  
		 \textbf{Concrete}:
		 \begin{itemize}
		 	\item Changing the steady-state by increasing the open-loop gain.
		 	\item Untouch the transient response by untouch the root locus in the neighbourhood of dominant closed-loop poles:
		 	\begin{description}
		 		\item [a)] poles and zeros close to each other;
		 		\item [b)] poles and zeros close to the origin;
		 		\item [c)] angle of the compensator must be small, $\angle(C(s))<5^{\circ}$.
		 	\end{description}
		 \end{itemize}
\end{frame}

\begin{frame}
	\frametitle{Lag compensators}
		$C(s)=K_c \beta\frac{Ts+1}{\beta Ts+1}= K_c\frac{s+\frac{1}{T}}{s+\frac{1}{\beta T}}$ with $\beta>1$\\
		\begin{itemize}
			\item Choose poles and zeros close together
			\item Choose $K_c$ close to 1 (if $K_c$ is exact 1, the transient response won't change)
			\item Take $\beta$ as large as possible
			\item Take T as large as possible 
		\end{itemize}
		\vspace{3mm}
		
		\textbf{Remark}: by the choice of $\beta$ and T, we have to take account of the reality (the physical realisation), so there is a limitation on the values.  \vspace{3mm}
		
		The \textbf{downside} of the lag compensator: the settling time will increase because the pole and zero of the closed-loop system are close to the origin. 
\end{frame}

\begin{frame}
	\frametitle{Lag compensators}
		Steps to make a lag compensator (in cascade with the original system):
		\begin{enumerate}
			\item Draw the root locus of the uncompensated open loop system and search the dominant closed loop poles on the root-locus (from the specifications).
			\item The lag compensator is $C(s)=K_c \beta\frac{Ts+1}{\beta Ts+1}= K_c\frac{s+\frac{1}{T}}{s+\frac{1}{\beta T}}$.
			\item Calculate the static error constant. 
			\item Calculate the new static error constant if we achieve all the specifications.
			\item From this difference we make the lag compensator. Now, we can see the pole and the zero of the compensator. 
		\end{enumerate}
		\textbf{Remark}: we assume that the lag compensator achieved the transient response specifications. If not: make a lead-lag-compensator.
\end{frame}

\begin{frame}
	\frametitle{Lag compensators}
		\begin{enumerate}
			\setcounter{enumi}{5}
			\item Draw the root locus of the compensated closed-loop system. Determine the locations of the dominant closed loop poles on the root-locus. 
			\item Adjust $K_c$ from the magnitude conditions such that the closed-loop poles lie at the desired location. ($K_c\approx 1$)
		\end{enumerate}
		\vspace{3mm}
		
		\underline{Some notes}:
		\begin{itemize}
			\item The ratio of the value of gain required in the specifications and the gain found in the uncompensated system is equal to the ratio between the distance of the zero from the origin and that of the pole from the origin.
			\item If the angle of the lag compensator is very small, then the original root loci and the new root loci are almost equal.
			\item In some circumstances can be used both a lead or a lag compensator. 
		\end{itemize}
\end{frame}

\begin{frame}
	\frametitle{Lag-lead compensators}
		\underline{General}:\\
		Lead compensators: 
		\begin{itemize}
			\item speeds up the response;
			\item increase stability of the system.
		\end{itemize}
		Lag compensators:
		\begin{itemize}
			\item improves steady-state accuracy;
			\item reduces the speed of the response.
		\end{itemize}
		Lag-lead compensator:
		\begin{itemize}
			\item If both transient response and steady-state must be improved.
			\item When 1 global component is more economical then both a lead and a lag component.
		\end{itemize}
		The lag-lead compensator has the advantages of both compensator. It has 2 poles and zeros, so the system has order 2.
\end{frame}

\begin{frame}
	\frametitle{Lag-lead compensators}
	The lag-lead compensator C(s) has this form: \\
	$C(s)=K_c\frac{\beta (T_1s+1)(T_2s+1)}{\gamma (\frac{T_1}{\gamma}s+1)(\beta T_2s+1)}=K_c(\frac{s+\frac{1}{T_1}}{s+\frac{\gamma}{T_1}})
	(\frac{s+\frac{1}{T_2}}{s+\frac{1}{\beta T_2}})$ with $\beta>1$ and $\gamma>1$.\vspace{4mm}
	Notes:
	\begin{itemize}
		\item The value $\beta T_2$ may not be to large, it must be fysical realizable.
		\item  There are two cases of this type compensator: $\gamma\neq \beta$ or $\gamma= \beta$.
	\end{itemize}
\end{frame}

\begin{frame}
	\frametitle{Lag-lead compensators}
	Case 1:$\gamma\neq \beta$,  the design process is a combination of the design of the lead compensator and that of the lag compensator:
	\begin{enumerate}
		\item Determine the location of the closed-loop poles (from the specifications)
		\item The lead part of the compensator must contribute the angle $\phi$: $\phi=\angle$(desired pole)$-\angle$(uncompensated open-loop system)
		\item 
		\begin{itemize}
			\item Take $T_2$ as large as possible.
			\item Determine $T_1$ and $\gamma$ such that:\\ 
			$\angle(\frac{s_1+\frac{1}{T_1}}{s_1+\frac{\gamma}{T_1}})=\phi$
			\item Determine $K_c$ from the condition:\\
			$|K_c\frac{s_1+\frac{1}{T_1}}{s_1+\frac{\gamma}{T_1}}G(s)|=1$ with G(s) the open loop transfer function.
		\end{itemize}
	\end{enumerate}
\end{frame}

\begin{frame}
	\frametitle{Lag-lead compensators}
	Case 1: $\gamma\neq \beta$,  the design process is a combination of the design of the lead compensator and that of the lag compensator:
	\begin{enumerate}
		\setcounter{enumerate}{3}
		\item Determine $\beta$ if $K_v$ is given:
		$K_v=lim_{s\to0}sC(s)G(s)=
		lim_{s\to0}sK_c\frac{\beta (T_1s+1)(T_2s+1)}{\gamma (\frac{T_1}{\gamma}s+1)(\beta T_2s+1)}=
		K_c(\frac{s+\frac{1}{T_1}}{s+\frac{\gamma}{T_1}})
		(\frac{s+\frac{1}{T_2}}{s+\frac{1}{\beta T_2}})G(s)=
		lim_{s\to0}sK_cG(s)\frac{\beta}{\gamma}$
	\end{enumerate}
\end{frame}




