\section{Introduction}

\begin{frame}
	\frametitle{Definitions}
		The main objective of this chapter: design and compensation of single-input-sigle-output linear time-invariant control systems.
		\begin{itemize}
			\item Compensation: the modification of the system dynamics to satisfy the given specifications.
			\item Specifications (transient response and steady-state requirements): given before the design.
			\item Design by root-locus method: making a new root locus by adding poles and zeros to the system's open-loop transfer function.
			\item Compensator: an other system inserted in parallel or in cascade with the system for the purpose of satisfying the specifications of the original system (e.g. lead, lag, lag-lead compensator's or PID controllers).
		\end{itemize}
\end{frame}

\begin{frame}
	\frametitle{Compensators}
	A sinusoidal input is applied to the input of a network. We got a:
	\begin{itemize}
		\item lead network: if the steady-state output has a phase lead.
		\item lag network: if the steady-state output has a phase lag.
		\item lag-lead network: if we have phase lag and phase lead in the output but in different frequentcy regions (lag when the input has low frequency and lead in high frequentcy). 
	\end{itemize}
	The amount of lag/lead is a function of the frequency. \vspace{4mm}
	
	A compensator with characteristic of a lead network, lag network, or lag-lead network is called a lead compensator, lag compensator, or lag-lead compensator. 
	
	\textbf{Remark}: with trail and error we find the optimal compensator
\end{frame}

\section{General design}

\begin{frame}
	\frametitle{Controllers}
		Scetching the problem:
		\begin{enumerate}
			\item We got a plant that not achieve all the specifications (and we cannot change the parameters of the original system).
			\item We have to change other parameters such that the system will achieve all the specifications.
			\item These other parameters can be changed by changing the root loci of the closed-loop system. 
			\item So we search the root loci of a compensator such that the overall system achieve the specifications. 
			\item We let the original system interact (cascade or parallel) with the compensator.
		\end{enumerate}
		\vspace{3mm}
		
		\textbf{Remark}: we discuss only continu time systems. 
\end{frame}

\begin{frame}
	\frametitle{Root locus approach}
	\underline{The method}:\\
	Given: the root loci of the open-loop system (this are the parameters).\\
	Now, the method consist of graphical determining the root loci of the closed-loop system.\vspace{3mm}

	\underline{Example}: 
	We want a certain gain of a system:
	\begin{itemize}
		\item The system is not stable at that gain.
		\item We have to change the root loci (and make the system stable) by designing a compensator
		\item Now, we got different root loci and the system is stable at the certain gain. So, the specifications are achieved.
	\end{itemize}
	\vspace{2mm}
	
	In \textbf{general}: we want to have the root loci of the system on the good locations.
\end{frame}













\begin{frame}
	\frametitle{Addition of poles/zeros}
	 \begin{block}{Addition of poles}
	 	The addition of a pole to the open-loop transfer function has the effect of pulling the root locus to the right, tending to lower the system's relative stability and to slow down the settling of the response.
	 \end{block}
	 \begin{block}{Addition of zeros}
	 	The addition of a zero to the open-loop trans- fer function has the effect of pulling the root locus to the left, tending to make the system more stable and to speed up the settling of the response.\\
	 	It also increase:
	 	\begin{itemize}
	 		\item the degree of anticipation into the system
	 		\item the speed of the transient response
	 	\end{itemize}
	 \end{block}
\end{frame}

\section{Lead copensations}

\begin{frame}
	\frametitle{Lead compensators}
		Steps for creating a compensator: 
		\begin{enumerate}
			\item Determine the desired location for the dominant closed-loop poles (from specifications). 
			\item If the adjustment of the gain connot yield the desired closed loop poles, then calculate the angle deficiency $\phi$. This deficiency has to by contributed by the lead compensator.
			\item We take a compensator of this form: $C(s)=K\frac{s+\frac{1}{T}}{s+\frac{1}{\alpha T}}$ with $0<\alpha<1$. 
			\item $\alpha$ and T must be created such that there isn't a angle deficiency. If this already be done, then take $\alpha$ as graet as possible. K can be found by the requirements of the open loop gain.
		\end{enumerate}
\end{frame}

\section{Lag compensation}

\begin{frame}
	\frametitle{Lag compensators}
	When the system unsatisfactory the steady state (but can satisfy the transient response)\\
	
	We go increase the open loop gain, while we try to untouch the transient response (concreet: don't change the root locus in the neigbourhood of the dominant closed-loops, but increase the open loop gain)\\
	
	How to fix a small changing of the root loci -> the angle contribution of the lag compensator should be not larger than 5° (how? we place the pole and zero of the lag network close togheter and close to the origin of the s-plane)\\
	
	$C(s)=K\frac{s+\frac{1}{T}}{s+\frac{1}{\beta T}}$ with $\beta>1$ beta can e large if the poles and zeros are close to the origin, T has to be taken large.
\end{frame}