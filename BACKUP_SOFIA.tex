\documentclass{document}
\begin{document}
	\section{H1}
		\subsection{Videos}
		\begin{itemize}
			\item Control is everywhere around us. This video shows examples of control and the importance of feedback.\\
			https://youtu.be/C221sI1W9Gk
			\item On the opening day of the Millenium Bridge in London (2000), approximately 90 000 people walked across the bridge (more than 2 000 at the same time). Because of the large number of people, a lot of them walked with the same rithm. This caused the bridge to wobble a few millimeters. Due to this swinging, people started to walk with these movements to keep their balance. This positive feedback caused the bridge to swing more and more. 
			After three days, the bridge was closed to adjust the bridge with shock absorbers and re-opened in 2002.\\
			https://youtu.be/eAXVa__XWZ8
			\item This movie shows spontaneous neuronal activity in the optic tectum of a zebrafish larva at 3 days postfertilization. Neuronal activity is visualized as increased fluorescence intensity of a calcium sensor (GCaMP7a), which is expressed in the neurons. Intracellular calcium concentration increases because of the calcium influx upon depolarization of the neurons. This movie shows spontaneous neuronal activity (that is, no visual stimulus was given in this recording except the excitation light for fluorescence imaging). It is replayed 5x real time.\\
			https://youtu.be/_rGEkYfQVwY
			\item When a drumstick hits a cymbal, it can be approximated by applying an impulse to the system. After some time, the distorition is stabilized. This effect is shown at 1000 frames/sec. \\
			https://youtu.be/kpoanOlb3-w
			\item Pilot making a risky maneuvre by almost landing and ascending again.\\
			https://youtu.be/gGnyWgXnZ6g
			\item In this video, 32 metronomes are all started at a different moment and are not in sync.
			When the arm of any metronome hits the side, it exerts a force on the blue platform. Normally friction would make that unnoticeable. But this platform is set up on rollers so that it can move from side to side.
			When any two metronome arms hit, their forces on the platform either cancel out or add together (feedback), depending on how out of or in sync they are. Any arms that are out of sync will experience a force in the opposite direction that inches them closer to the pack. 
			Eventually all 32 arms find the same rhythm and sync up.\\
			https://youtu.be/5v5eBf2KwF8
			\item The video explains in simple words and with a little example what automation is, why it is important and where it can be found.
			The video was made by a team of young researchers from the Dipartimento di Elettronica, Informazione e Bioingegneria of the Politecnico di Milano, coordinated by a young director.
			[IEEE CSS Video Clip Contest 2014 Submission]
			https://youtu.be/XJLMW6l303g
			\item When a guitar player strikes a string, it starts vibrating. These vibrations are picked up by the guitar pickups, which send it to the amplifier. If the guitarist points his guitar to the amplifier, then the sound coming from the amplifier resonates with the original vibration, which has the same frequency. This creates a positive feedback loop between the guitar string and the amplifier, resulting in that typical feedback sound you hear in the video above.\\
			https://youtu.be/luURyH9fzhk
			\item Feedback is everywhere. This video shows some examples of positive and negative feedback loops in biology. 
			https://youtu.be/CLv3SkF_Eag
			\item Impossible for humans, but effortless for automatic control.
			Flying six vehicles simultaneously and coordinating their motion to music. An example of the power of automatic control.
			[IEEE CSS Video Clip Contest 2014 Submission] \\
			https://youtu.be/NRL_1ozDQCA
			\item The video shows fully automated driving and drifting of a sports car. The vehicle is capable of tracking a course reference signal within a few centimeters. At the same time the vehicle stabilizes either its velocity (first part of the video) or its slip angle (second part).
			[IEEE CSS Video Clip Contest 2014 Submission]\\
			https://youtu.be/1FVglbJZ_tg
			\item The Cubli is a 15 × 15 × 15 cm cube that can jump up and balance on its corner. Reaction wheels mounted on three faces of the cube rotate at high angular velocities and then brake suddenly, causing the Cubli to jump up. Once the Cubli has almost reached the corner stand up position, controlled motor torques are applied to make it balance on its corner. In addition to balancing, the motor torques can also be used to achieve a controlled fall such that the Cubli can be commanded to fall in any arbitrary direction. Combining these three abilities -- jumping up, balancing, and controlled falling -- the Cubli is able to 'walk'. It is a beautiful example of control.\\
			https://youtu.be/n_6p-1J551Y
			\item A platform for planar magnetic manipulation (Magman) consisting of an array of 16 coils is modeled and controlled in real-time from Matlab/Simulink. Magman manipulates with rolling steel balls in a way that would not be possible without feedback. After the intro, the video shows two means of measuring position, identification of the system, modeling results and design and testing of the control system that can be controlled remotely by iPad.
			[IEEE CSS Video Clip Contest 2014 Submission]\\
			https://youtu.be/AhS_2gU1qW0
			\item Another example of control and feedback.
			This robot can predict the orbit of the badminton shuttle and return it.\\
			https://youtu.be/LSax71cn6A4
			\item Raffaello D'Andrea: The astounding athletic power of quadcopters.
			These quadcopters can balace an inverted pendulum and a glass full of water, predict the orbit of a ball and return it and stabilize their position.\\
			https://youtu.be/w2itwFJCgFQ
			\item Autonomous Car: https://youtu.be/RY93kr8PaC4
			\item Control-Theoretic Swarm Joysticks: https://youtu.be/TgQ6f0xFwbE
			(Published on Aug 12, 2014 [IEEE CSS Video Clip Contest 2014 Submission] This video highlights the work on multi-agent robotics at the GRITSLab at the Georgia Institute of Technology. By drawing inspiration from the world around us, different ways of interacting with the robots are discussed, with control theory playing a key role for making the human-swarm interactions happen)
			\item UAV Calligraphy: https://youtu.be/zXLmulzKWvQ
			(Clip for IEEE VSS Video Contest 2014)
		\end{itemize}
		\subsection{Brainteasers}
		\begin{itemize}
			\item The order of a system is:\\
				A. The number of input variables, B. The number of output variables , \textbf{C. The number of state variables}, D.​The sum of the number of input variables and the number of output variables
			\item In a dynamic system: \\
			A. The output only changes if the input changes. B. The output changes with time. \textbf{C.The output changes with time if the system is not in a state of equilibrium.}
			\item In a dynamic system: \\
			\textbf{A. The current outputs depend on the past inputs.} B. The current outputs only depend on the current inputs. C.The current outputs do not depend on the input.
			\item A controller is responsible for:\\
			A. Computing the difference between the reference signal and the output
			\textbf{B. Altering the operating conditions of a dynamical system}
			C. Computing the outputs using the given inputs
			\item If there are a lot of noise disturbances, you best use: 
			A. An open loop system
			B. \textbf{A closed loop system}
			C. 	Either of the above
		\end{itemize}
		
	\section{H2}
		\subsection{Video}
		Rodolphe Sepulchre is a researcher in control engineering. His research lab at Cambridge University
		aims at unfolding the mechanisms that enable robustness, performance, and adaptation in artificial
		and natural systems. Neuronal circuits in the brain have become its main source of inspiration to explore the simplifying role of feedback and nonlinearity in the multi resolution organization of behaviors.\\
		https://youtu.be/OXoJJX3GOaQ
		\subsection{Brainteasers}
		\begin{itemize}
			\item A distributed parameter system is a system described by: \\
			A. Ordinary Differential Equations (ODEs)
			\textbf{B. Partial Differential Equations (PDEs)}
			C. Algebraic Equations
			\item Let's consider the following system,
			Md2x(t)dt2+F(t)dx(t)dt+Kx(t)=u(t)
			where u(t) is the system input, x(t) is the system output, and M, F(t) and K are the parameters of the system. Which of the following statements are correct?\\
			It is a
			- linear system
			- distributed parameter system
			- time-varying system
			​- continuous-time system
			It is a
			- linear system
			- lumped parameter system
			- time-varying system
			​- discrete-time system
			It is a
			- linear system
			- lumped parameter system
			- time-varying system
			​- continuous-time system
			It is a
			- nonlinear system
			- distributed parameter system
			- time-varying system
			​- continuous-time system
			It is a
			- nonlinear system
			- lumped parameter system
			- time-invariant system
			​- continuous-time system
			It is a
			- nonlinear system
			- lumped parameter system
			- time-varying system
			​- continuous-time system
			\item Let's consider the following system,
			y[k]=5y[k−1]+8y[k−2]u[k]−10sin(0.5y[k−3])
			where u[k] is the system input and y[k] is the system output. Which of the following statements are correct?
			It is a
			- linear system
			- lumped parameter system
			- time-invariant system
			​- continuous-time system
			It is a
			- linear system
			- lumped parameter system
			- time-invariant system
			​- discrete-time system
			It is a
			- nonlinear system
			- lumped parameter system
			- time-invariant system
			​- continuous-time system
			It is a
			- nonlinear system
			- lumped parameter system
			- time-invariant system
			​- discrete-time system
			It is a
			- nonlinear system
			- lumped parameter system
			- time-varying system
			​- discrete-time system
			It is a
			- nonlinear system
			- distributed parameter system
			- time-invariant system
			​- discrete-time system
			\item Let's consider the following system,
			∂C∂t=−v∂C∂x−k0Ce−ERT∂T∂t=−v∂T∂x+GrCe−ERT+Hr(TJ−T)
			where C(x,t) and T(x,t) are the output variables, TJ(x,t) is the system input, and v, E, R, Hr and Gr are the parameters of the system. Which of the following statements are correct?
			It is a
			- linear system
			- lumped parameter system
			- time-invariant system
			​- causal system
			It is a
			- nonlinear system
			- distributed parameter system
			- time-varying system
			​- non-causal system
			It is a
			- linear system
			- distributed parameter system
			- time-invariant system
			​- causal system
			It is a
			- nonlinear system
			- distributed parameter system
			- time-varying system
			​- causal system
			It is a
			- nonlinear system
			- lumped parameter system
			- time-varying system
			​- non-causal system
			It is a
			- nonlinear system
			- distributed parameter system
			- time-invariant system
			​- causal system
		\end{itemize}
			
		\section{H3}
			\subsection{videos}
			\begin{itemize}
			\item A short introduction, showing the beauty of matematics.
			https://youtu.be/-7vOWe0mpXQ
			\item An example of an inverted pendelum. The goal is to keep the pendulum stable. This is an example of a control problem. \\
			https://youtu.be/15DIidigArA
			\item An introduction to non-linear-time systems. It explains how to linearize the system.\\ https://youtu.be/qBFWbiOJPBA
		\end{itemize}
		\subsection{brainteasers}
		\begin{itemize}
			\item Given a non-linear dynamic system with inputs u and output y, described by the differential equation y˙=u˙+uy+y. Suppose you have succesfully linearized the non-linear term uy as a linear function of u and y. What is the minimum order of a realization of this system?
			A. 1st Order
			\textbf{B. 2nd Order}
			C. 3th Order
			\ We continue with the dynamical system from the previous question. Now find a State Space Representation of this system with the minimal order. Which of the following answers is a correct linearization arround \vec{x}_e,u_e?\\
			A. $\frac{d \Delta x}{dt} = \begin{bmatrix} 0 & 0 \\ 1 & u_e + 1 \\ \end{bmatrix} \begin{bmatrix} \Delta X _1 \\ \Delta X_2 \end{bmatrix} + \begin{bmatrix} 1 \\ x_{2,e} \end{bmatrix} \Delta U$
			B. $\frac{d \Delta x}{dt} = \begin{bmatrix} u_e & 1 \\0 & 0 \\ \end{bmatrix} \begin{bmatrix} \Delta X _1 \\ \Delta X_2 \end{bmatrix} + \begin{bmatrix} x_{e,1} \\ 1 \end{bmatrix} \Delta U $
			C.$\frac{d \Delta x}{dt} = \begin{bmatrix} 0 & 0 \\ 1 & 1 + u_e\\ \end{bmatrix} \begin{bmatrix} \Delta X _1 \\ \Delta X_2 \end{bmatrix} + \begin{bmatrix} 1 \\ 1 \end{bmatrix} \Delta U$
			\item Given the image of the electrical circuit above. What is the minimal order of this dynamic system?\\
			A. 1ste order system
			B. 2nd order system
			C. 3th order system
		\end{itemize}
		\section{H4}
		\subsection{Videos}
		\begin{itemize}
				\item A video explaining the process of sampeling a digital signal. A digital signal is an example of discrete time signal.\\
				https://youtu.be/BNVVq-iVPy8
				\item A short introduction to complex numbers by Khan acadamy. Note that this movie is made by a mathimatican so i -> j.
				$i^2 = - 1 \Rightarrow i = \sqrt{- 1}$\\
				$z_1 = a + b j\\ z_2 = c+dj$
				$z_1 + z_2 = (a+c) + (b+d)j\\ z_1 - z_2 = (a-c) + (b-d)j\\ z_1*z_2 = (ac-bd) + (ad+bc)j\\ \frac{z_1}{z_2} = \frac{(ac+bd)-(ad+bd)j}{c^2+d^2}$
				https://youtu.be/kpywdu1afas
				\item Explaining how to go from a block diagram to state space model. Note that this movie is about a continu time system and the integrators are replaced by (1/s), which is the same.\\
				https://youtu.be/ifbAijeblKE
				\item Fibonacci sequence in nature.
				1,1,2,3,5,8,13,21,34,55, ... \\https://youtu.be/P0tLbl5LrJ8
				\item Explaining the convolution in discrete time. $\begin{align} y[k]& = u[k]*v[k]\\ &= \sum_{-\infty}^{\infty} u[i]v[k-i] \end{align}$ https://youtu.be/T-OwCIOlbm0
				\item Short introduction to Z-transform. $X(z) = \sum\limits_{k=0}^{+ \infty} x[k] z^{-k}$ 
				https://youtu.be/eeCiXEaVYcg
		\end{itemize}
		\subsection{Brainteasers}
		\begin{itemize}
			\item 	Determine the constant of Bart. The constant of Bart can be found by taking the limit of the ratio of two consecutive numbers from Barts sequence. A number in Barts sequence x[k] is the sum of 5 times x[k-1] and 3 times x[k-2]?\\
			A. $ß = {5 + \sqrt{29} \over 2}$
			B. $ß = {5 + \sqrt{39} \over 2}$
			C. $ß = {5 + \sqrt{3 9} \over 4}$
			D. $ß = {5 + \sqrt{29} \over 4}$
			\item The A-matrix in the state-space representation of a minimal LTI with homogeneous difference equation y[k+6]-y[k] = 0 has exactly 2 different eigenvalues.\\
			A. True
			B. False
			\item The A-matrix in the state-space representation of a minimal LTI with homogeneous difference equation y[k+6]-y[k] = 0 has exactly 2 different eigenvalues.\\
			A. True
			B. False
			\item The A-matrix in the state-space representation of a minimal LTI with homogeneous difference equation y[k+6]-y[k] = 0 has exactly 2 different eigenvalues.\\
			A. True
			B. False
			\item A system S with 3 inputs and 1 output and H(z) = $\begin{bmatrix} \frac{1}{z+1} & \frac{1}{(z+1)(z+2)} & \frac{1}{z+2} \end{bmatrix}$ . Than this system can be constructed using exactly 3 delay elements and a number of scalors and adders. 
			A. True
			B. False
			\item A discrete time system with a pole in 0.9 and a zero in - 0.9 is a
			A. low-pass filter
			B. high-pass filter
			C. none of the above
			\item If two systems with impulse response h[k] and g[k] are connected in parallel, then the equivalent system has an impulse response which is the sum of both. The same thing is true for the transfer function.
			A. True
			B. False
			C. None of the above
			\item If two systems with impulse response h[k] and g[k] are connected in series, then the equivalent system has an impulse response which is the product of both. The same thing is true for the transfer function.\\
			A. True
			B. Only for the transfer function
			C. None of the above
		\end{itemize}
	\section{H5}
		\subsection{Videos}
		\begin{itemize}
			\item 	This video shows a micro-mixing reactor, which is a perfect example of a continuous-time system. Continuous-time systems views variables as having a particular value for potentially only an infinitesimally short amount of time. Between any two points in time there are an infinite number of other points in time. In this application, the variable are e.g. solution injection A, solution injection B, solution that flows out of the reactor, ....\\
			https://youtu.be/qfyTW4nqYQI
			\item This easy to understand video derivates the Laplace Transform of  f(t)=et by using the definition of the Laplace Transform: L{f(t)}=∫∞0e−stf(t)dt which is given in the slides of this chapter. It is very useful to know the Laplace Tranform of commonly used functions, this way you don't have to calculate the integral every single time. The first time, it is of course necessary to use the definition of Laplace Transform, but once you have the result, you can just use this in applications. As the video shows: L{et}=1s−1for s∈(0,∞).\\
			https://youtu.be/wI6ki-_79nc
			\item This video of Khan Academy explains how you can solve linear differential equations by using the Laplace Transform. Next steps must be followed in order to obtain a correct solution:
			Take the Laplace Transform of the entire equation;
			Make use of the property of the Laplace Transform of the derivatives: L{y˙(t)}=sL{y(t)}−y(0) and fill in the given start conditions;
			Group the Laplace-terms and the constant terms;
			Simplify the equation until L{y(t)} is on the left-hand side of the equality sign;
			Apply a partial fraction decomposition and the Inverse Laplace Transform on the right-hand side of the equation obtained by the previous step.\\
			https://youtu.be/3uYb-RhM7lU
			\item A video that explains how to use transfer functions in MATLAB. MATLAB is an extremely powerful tool to solve mathematical problems and consequentely also very useful to solve problems concerning systems.\\
			https://youtu.be/7-IsUm0hDMg
			\item This video explains how to obtain the transfer function starting from a linear state-space model of a system. Starting from the time domain, we take the Laplace transform of the state-space model, next we eliminate X(s) in the expression of Y(s). This leads to the following expression: Y(s)=(C(sI−A)−1B+D)U(s).\\
			https://youtu.be/bgVEa1E5I04
			\item This video shows the swaying and finally the collapse of the original Tacoma Narrows Bridge in the state of Washington. This bridge was built with girders of carbon steel anchored in huge blocks of concrete and shortly after the construction, the bridge would appear to sway and buckle in relatively mild windy conditions, promting her nickname: Galloping Gertie. The failure of the bridge occurred when a never-before-seen twisting mode occurred, from winds at a mild 64 km per hour. The twisting mode is also called the torsial vibration mode and was provoked by aeroelastic fluttering. It causes the left side of the roadway to go down while the right side of the roadway rises with the center line of the road remaining still. 
			Fluttering is a physical phenomenon in which several degrees of freedom of a structure become coupled in an unstable oscillation driven by the wind. This movement inserts energy to the bridge during each cycle so that it neutralizes the natural damping of the structure; the composed system (bridge-fluid) therefore behaves as if it had an effective negative damping (or had positive feedback), leading to an exponentially growing response. In other words, the oscillations increase in amplitude with each cycle because the wind pumps in more energy than the flexing of the structure can dissipate and finally drives the bridge toward failure due to excessive deflection and stress. 
			A system is underdamped when its damping ratio satisfies the following contition: 0<ζ<1. In case of the Tacoma Narrows Bridge, the damping ratio decreases: when ζ becomes less than 1, the systems becomes underdamped, when ζ further decreases and reaches 0, the systems becomes undamped. The wind speed that causes the beginning of the fluttering phenomenon (ζ=0) is known as the flutter velocity. Fluttering occurs even in low-velocity winds with steady flow.\\
			https://youtu.be/j-zczJXSxnw
			\item A critically damped systems return as fast as possible to its equilibrum position without oscillating. A gun is made to be critically damped so that it returns to the neutral position in the shortest amount of time between firing. For something to be critically damped, it is required that the systems has two equal poles or a damping ration of ζ=1. \\
			https://youtu.be/EHCmibM3CWs
			\item This video tackles an example of an overdamped system. A system is overdamped when its damping ration satisfies the following condition: ζ>1. This RLC-circuit is proven to be overdamped by calculating the damping ratio. The video shows how to address practical problems concerning second order systems.\\
			https://youtu.be/DKyJDQrx1pw
		\end{itemize}
	\subsection{Brainteasers}
	\begin{itemize}
		\item 	Given is a continuous-time system H with a transfer function $H(s)=\frac{(s−3)(s+2)}{(s+4)(s+5)}$. When the zero in s=−2 on the negative axis changes in the direction of 0, the amplification of a constant in steady-state remains the same.\\
		A. True
		B. False
		\item Two linear, time invariant systems are connected in series. For a periodic input $ u(t)=\sin(\alpha t)$, is $yss=|H1(jα)||H2(jα)|\sin(αt+∠\angle H1(jα)+∠a\angle (jα))  $the steady state output ? 
		A. True
		B. False
		C. I don't know.
		\item We have found the step-response of a physical system with 1 input and 1 output.This response converges to a constant value.Can we conclude that the system is internally stable? 
		A. True
		B. False
		C. I don't know.
		\item When the system shown in Figure (a) is subjected to a unit-step input, the system output responds as shown in Figure(b). Determine the values of K and T from the response curve
	\end{itemize}
	\section{H6}
		\subsection{Videos}
		\begin{itemize}
			\item A video that shows very graphically how the response of a system can change as the frequency of the input signal varies.
			This is what we mean by the frequency response of system, H(jω). Remark however that in this video the frequency f is used, with the unit Hz. We always use ω=2πf in rad/s.
			So you can see the response of the system (the salt in this case) changes as the frequency of the input signal (the oscillating plate) changes. This is exactly what we try to express in bode plots and nyquist plots (see later this lecture).\\
			https://youtu.be/wvJAgrUBF4w
			\item A humorous video explaining the concept of resonance. \\
			https://youtu.be/urYWaHfel6g
			\item 
		\end{itemize}
		\subsection{Brainteasers}
		\begin{itemize}
			\item What is the steady state output of this system, if we use an input signal: $2sin(10t+1.25)?$
			A. $4.9sin(10+2.86)$
			B. $9.8sin(10t−1.72)$
			C. $9.8cos(10t+1.29)$
			D. $9.8cos(10t−4.68)$
			\item What is the transfer function corresponding to this bode plot?
			A. $H(s)=\frac{100(s+1)}{(s+100)(s+0.1)}$
			B. $H(s)=\frac{10(s+1)}{(s+100)(s+0.1)}$
			C. $H(s)=\frac{s}{(s+100)(s+0.1)}$
			D. $H(s)=\frac{10(s−1)}{(s+100)(s+0.1)}$
			\item What is the transfer function corresponding to this bode plot?
			A. $H(s)= \frac{100(s+1)(s+10)}{(s+100)^2(s+0.01)}$
			B. $H(s)=\frac{10(s+1)(s−1)}{(s+100)^2(s+0.01)}$
			C. $H(s)=\frac{100(s+1)(s+1)}{(s+100)^2(s+0.01)}$
			D. $H(s)=\frac{100(s+1)(s−1)}{(s+100)^2(s+0.01)}$
			\item You can see the bode plot of a second order system. What is the transfer function that corresponds the most to this bode plot? Tip: look at the form of second order systems under the slides of resonance.\\
			A. $H(s)=\frac{100}{\frac{s^2}{100}+\frac{2s}{100}+10$
			B. $H(s)=\frac{100}{s^2+20s+100}$
			C. $H(s)=\frac{100}{s^2+2s+100}$
			D. $H(s)=\frac{100}{s^2−2s+100}$
			\item You can see a point indicated on this nyquist plot of a system. Where can we find this point in the bode plot? At frequency -10? At frequency 10?
			\item If two continuous-time systems have the same magnitude plot, their transfer function is also the same.\\
			A. True
			B. False
			\item A continuous-time systems has two poles. Can the phase plot descend more than -180 degrees?\\
			A. Yes
			B. No
			\item On a bode plot, we can read the exact DC gain, this is the gain for frequency zero, $|H(j0)|$.\\
			A. True
			B. False
			\item If $H(j\omega)=0$ the magnitude plot will go to $-\infty$. But for negative values the magnitude plot is undefined, as you would take the logarithm of a negative value.
			A. True
			B. False
			\item A system with transfer function $H(s)=\frac{s+100}{s(s+10)}$ is given. What does the nyquist plot do at frequency 0 if we go from $−\infty$ to $\infty$?\\
			A. It is simply a point at $\infty$
			B. It is simply a point at $-j\infty$
			C. It jumps from $-j\infty$ to $j\infty$
			D. It jumps from $j\infty$ to$-j\infty$
		\end{itemize}
	\section{H7}
	\subsection{Videos}
	\begin{itemize}
		\item Relationship of Laplace transform s-plane to Z-plane of the Z transform.
		How does the left-half s-plane winds up inside the unit circle of the z-plane? And how does the origin of the z-plane actually corresponds to the point-at-minus-infinity of the s-plane rather than its origin?
		
		Relationship between Laplace transform s-plane to Z-plane of the Z transform ( Z = exp(s.Ts), where Ts is the sampling period) is illustrated with the help of a strip of the s-plane (for frequencies from f = -Fs to f = Fs, where Fs = 1/Ts is the sampling frequency).
		The mapping Z = exp(s.Ts) = exp(sigma . Ts) . exp(j 2.pi. f . Ts)
		The exp(j 2.pi. f . Ts) factor identifies the horizontal line f = -1/Ts with the horizontal line f = 1/Ts, thus creating a cylinder. Then the factor exp(sigma . Ts) turns that cylinder into a funnel-like shape. When that funnel is viewed from the front position, we clearly see the familiar Z-plane.
		What is not shown is the entire s-plane, because the s-plane will be folded onto the strip shown in the video (by frequency aliasing). This is left as an exercise to imagine.\\
		https://youtu.be/4PV6ikgBShw
		\item Aliasing occurs when your measurement frequency is not fast enough to accurately capture the measured event. In general, you want your measurement device to be one order of magnitude faster than the event being measured.
		This video shows an example of aliasing with a spinning wire.\\
		https://youtu.be/A-19SxqZ8Qs
		\item Jim Janossy uses Audacity to demonstrate how the sampling rate for sound files affects the fidelity of the sound as reproduced. 
		If the sampling rate is low, information will be lost. If we increase the sampling rate, the sound will continuously get clearer. \\
		https://youtu.be/ZoyX7-SksqE
		\item Aliasing
		(Aliasing is when the sample rate that is less than the actual frequency of something happening creates an illusion of an alternate frequency. In this case, the sample rate of the camera (120 FPS) is much lower than the frequency of which the gas turbine is turning at, create an illusion that the gas turbine is turning much slower than it is in reality. At some point, the gas turbine looks as if it stops and eventually goes in reverse, because of aliasing.)
		Aliasing
		Sampling rate
		Shannon sampling frequency
		Discretization
		Frequencies and sound
		Digital audio
	\end{itemize}
	\subsection{Brainteasers}
	\begin{itemize}
		\item If $X(j\omega)$ is the Fourier transform (the spectrum) of $x(t)$ and the amplitude spectrum is $|X(j\omega)|$, then the functions $x(−t), x(t+8)$ and $\frac{x(t)+x(−t)}{2}$ all have the same amplitude spectrum as $x(t)$.
		A. True
		B. False
		\item One samples a signal x(t) with a frequency band [−f′2,f′2] with sampling frequence f with f′=1.5f and later reconstructs the signal with an ideal low-pass filter with bandwidth f/2. For all frequencies, the amplitude of the output differs from the amplitude of the input at the same frequency.
		A. True
		B. False
		\item One samples a signal x(t) with a frequency band $\begin{Bmatrix}
			−f′2,f′2\\
		\end{Bmatrix}$ with sampling frequence f with f′=f/2 and later reconstructs the signal with an ideal low-pass filter with bandwidth f/2. For all frequencies, the amplitude of the output differs from the amplitude of the input at the same frequency.
		A. True
		B. False
		\item Consider a continuous time system with transfer function $H(j\omega)=e^{-j\omega T}$ (which means a delay over a time interval T>0). It delivers an output of sines that all have the same amplitude as the sine on the input, and they are all displaced over the same phase. In other words, the amplitude and phase characteristics of this system are both constants.
		A. True
		B. False
		\item $x1(t)$ and $x2(t)$ are two band-limited continuous signals with Nyquist sampling frequencies $f1$ resp.$ f2$. The Nyquist sampling frequencies of the signals $x1(t)+x2(t)$, $x1(t).x2(t)$ and $x1(t)∗x2(t)$ are all$ f1+f2$.
		A. True
		B. 
		\item If $x(t)$ is a band-limited signal with bandwith B = 1000 Hz and Nyquist sampling frequency f, than f is the Nyquist sampling frequency of the following signals: $100x(t), x(t−5), x2(t)$ and $x(t)cos(t).$
		A. True
		B. False
	\end{itemize}
	\section{H9}
		\subsection{Videos}
		\begin{itemize}
			\item 	To obtain simulations of the continuous-time systems, a discrete-time equivalent of that system is required. This makes discretization of continuous-time systmes indispensable in today's world. This video shows several simulations of real-life objects. By using a discrete-time equivalent of the continuous-time system and a discrete-time model of the physical phenomenons the system is subject to, you can simulate the complete situation on a logical device, which is less expensive than making a real-life simulation (e.g. a car-crash test).\\
			https://youtu.be/Mo8rnhGNbnE
			\item Prewarping is a consequence of the bilinear rule. The enire jω-axis is compressed into the 2π-length of the unit circle, causing a frequency distortion. The video illustrates the mapping from the s-plane into the z-plane by the relation: $\omega a=\frac{2}{T_s}\tan(\frac{\omega T_s}{2})$, which derivation can be found in the slides of this chapter.\\
			\item The zero-order hold equivalent is a discretization method. The relation: Hzoh(z)=(1−z−1)Z\{\frac{H(s)}{s}\} given in the slides of this chapter, can be derived by substracting two step-responses. This video shows a complete derivation of this formula hence giving you a deeper apprehension of the method. A similar derivation can be made for the first-order hold equivalent.
		\end{itemize}
	\subsection{Brainteaser}
	\begin{itemize}
		\item Suppose I have a continuous-time system driven by continuous-time inputs and outputs. Suppose I sample the inputs and outputs with my favorite method. The relation between the inputs and outputs after discretization will still be linear time invariant.
		A. True
		B. False
		\item If we require a digital filter with a cut off of 100Hz and a sampling frequency of 625Hz, using the Tustin method with prewarping, the analog filter must have a cut-off of:
		A. 0.875 Hz
		B. 109.3702 Hz
		C. 100.2139 Hz
		D. 0.0018 rad/s
		\item We use the impulse invariant method to compute the discrete-time equivalents of the continuous-time system S and the continuous-time system T. The continuous-time system S has an impulse response h(t) and the discrete-time equivalent has an impulse response h[k]. The continuous-time system T has an impulse response g(t) and the discrete-time equivalent has an impulse response g[k]. 
		In this case, the system U, consisting of S and T in parallel, has in an impulse response h(t) + g(t) and a discrete-time equivalent with impulse response h[k] + g[k]. The system V, consisting of S and T in series, results in an impulse response h(t) * g(t) and a discrete-time equivalent with impulse response h[k] * g[k].\\
		A. True
		B. False
		\item We prefer the backward euler method instead of the forward euler method when we want to discretize a system with transfer function $H(s)=\frac{s^2+3}{s2+4s+3$}.\\
		A. True
		B. False
		\item 
	\end{itemize}
	\section{H10}
	\subsection{Videos}
	\begin{itemize}
		\item This video is a short introduction to the root locus method. The root locus of an (open-loop) transfer function H(s) is a plot of the locations (locus) of all possible closed-loop poles with proportional gain K and unity feedback. The method provides a visualization, of the roots of the denominator of the transfer function, which you can use to chose the gain K. Most of the time you cannot chose an arbitrary gain but you want your system to meet certain requirements. The most important is probably that you want your closed-loop system to be stable. However, it is not always possible to make your closed-loop system stable by only adjusting the gain, as the video shows. Sometimes you will have to add poles and zeros but then you're no longer working with a unity feedback. In this chapter we will only look at a closed-loop system with a proportional gain and unity feedback, other feedback controllers are discussed in the next chapters.\\
		\item The root locus method is often used to make the system, which you're designing, meet the necessary requirements. This video mentions three requirements: the damping ratio, the time for decay to half and the (natural) frequency. In the lecture slides more of these requirements and their test criteria are explained.
		\item This video shows how you can use the root locus method in MATLAB. It demonstrates the 'rlocus(system)'-function, the 'grid'-function and the 'rlocfind(system)'-function. It also indicates how you can make use of the SISOTOOL. These functions are also explained in the lectures, but this video will clarify them more.\\
		https://youtu.be/cEMbBCc0jUI 
		
	\end{itemize}
	\subsection{Brainteasers}
	\item The root locus of a system with transfer function: $H(s)=\frac{5(s+1)(s−1)}{(s+3)(s+5)}$, is completely situated in the right-half part of the s-plane.\\
	A. False
	B. True
	\section{H11}
	\subsection{Videos}
	\begin{itemize}
		\item The effect of feedback in the provision of electricity.\\ https://youtu.be/-8cM4WfZ_Wg
		\item A video explaining the Cauchy argument principle and the Nyquist stability criterion.\\
		https://youtu.be/sof3meN96MA
		\item Video explaining how to construct the Nyquist plot.\\ https://youtu.be/tsgOstfoNhk
		\item Video explaining phase and gain margins.\\
		https://youtu.be/ThoA4amCAX4
	\end{itemize}
	\subsection{Brainteaser}
	\begin{itemize}
		\item Consider the following open loop transfer function $H(s)=\frac{90}{s^2+9s+18}$ and corresponding Nyquist plot in the image above.
		The closed loop system is:\\
		A. Stable
		B. Unstable
		\item Consider the following open loop transfer function $H(s)=\frac{10(s+3)}{s^2−4}$ and corresponding Nyquist plot in the image above.
		The closed loop transfer function is:\\
		A. Stable
		B. Unstable
		\item Consider the following open loop transfer function $H(s)=\frac{80}{s3+6s2+11s+6}$ and corresponding Nyquist plot in the image above.
		The closed loop transfer function is:
		A. Stable
		B. Unstable
		\item Consider the following open loop transfer function $H(s)=\frac{20}{s3+5s2+6s}$ and corresponding Nyquist plot in the image above.
		The closed loop transfer function is:\\
		A. Stable
		B. Unstable
		\item Consider the following open loop system $H(s)= \frac{50(s+3)}{(s−3)(s2+2s+17)}$ and corresponding Nyquist plot in the image above.
		The closed loop system is:\\
		A. Stable
		B. Unstable
		\item Consider the following transfer function $H(s)=\frac{s^2−5s+6}{s^3+7s^2+14s+8}$. For the closed loop system to be stable, the Nyquist plot has to:\\
A. 		Encircle the origin twice in clockwise direction
		B. Encircle (-1,0) twice in clockwise direction
		C. Encircle the origin twice in counterclockwise direction
		D. Encircle (-1,0) twice in counterclockwise direction
		E. Have no net encirclements
		\item Consider the following transfer function $H(s)=\frac{s^2−9}{s^2−4}$. For the closed loop system to be stable, the Nyquist plot has to:\\
		A. Encircle the origin once in clockwise direction
		B. Encircle (-1,0) once in clockwise direction
		C. Encircle the origin once in counterclockwise direction
		D. Encircle (-1,0)  once in counterclockwise direction
		E. Have no net encirclements
	\end{itemize}
	
\end{document}

